\documentclass[prodmode,acmtecs]{acmsmall} % Aptara syntax
% Package to generate and customize Algorithm as per ACM style
\usepackage[ruled]{algorithm2e}
\renewcommand{\algorithmcfname}{ALGORITHM}
\SetAlFnt{\small}
\SetAlCapFnt{\small}
\SetAlCapNameFnt{\small}
\SetAlCapHSkip{0pt}
\IncMargin{-\parindent}
\usepackage{listing}
\usepackage{minted}
\usepackage{xspace}
\usepackage[colorinlistoftodos]{todonotes}
\newcommand{\rustname}{{\texttt{Rust}}}
\def \rust {\rustname{}\xspace}
\newcommand{\rustcname}{{\texttt{rustc}}}
\def \rustc {\rustcname{}\xspace}
\newcommand{\cname}{{\texttt{C}}}
\def \c {\cname{}\xspace}
\newcommand{\cppname}{{\texttt{C++}}}
\def \cpp {\cppname{}\xspace}
\newcommand{\mirname}{{\texttt{MIR}}}
\def \mir {\mirname{}\xspace}
\newcommand{\hirname}{{\texttt{HIR}}}
\def \hir {\hirname{}\xspace}
\newcommand{\llvmname}{{\texttt{LLVM}}}
\def \llvm {\llvmname{}\xspace}
\newcommand{\llvmirname}{{\texttt{LLVMIR}}}
\def \llvmir {\llvmirname{}\xspace}
\newcommand{\mlname}{{\texttt{ML}}}
\def \ml {\mlname{}\xspace}
\newcommand{\vecname}{{\texttt{Vec}}}
\def \vec{\vecname{}\xspace}
\newcommand{\projectname}{{\texttt{STOUT}}}
\def \name{\projectname\xspace}

\begin{document}
\title{\name: Safe Structure Splitting in a General Purpose Language}
\author{Jacob Bisnett}
\maketitle
\pagenumbering{arabic}
\section{Introduction}
\label{sec:intro}

The programming language \rust is fairly new, but already gained a considerable
following due to its placement as a safe and easy to use alternative to \c and
\cpp. \rust's primary advantage is its robust compile time checks and type
system. Among many of these are:
preventing two mutable references to an object from existing at the same time, 
statically determining when to free objects without reference counting or
garbage collection,
and not allowing references to outlive their objects.

Taken all together, \rust helps the user avoid data races and memory safety
errors by preventing programs that could contain them from compiling.
\rust does allow users to circumvent these rules with ``unsafe'' code,
but this unsafety must be explicitly declared and thus dangerous behavior can be
isolated.

While \rust is not the only language that provides this level of safety, it is
the only popular language that provides this level of safety with little to no overhead,
making it potentially as fast as \c and \cpp while avoiding many of the
pitfalls of the two languages.

\begin{figure}
  \centering
\begin{tabular}{r|c|c}
  Test&Rust&C\\
  k-nucleotide&5.30&6.46\\
  pidigits&1.75&1.73\\
  spectral-norm&2.01&1.98\\
  reverse-complement&0.45&0.42\\
  fasta&1.49&1.32\\
  mandelbrot&1.90&1.64\\
  n-body&13.08&9.56\\
  regex-redux&3.28&1.89\\
  binary-trees&4.27&2.39\\
\end{tabular}

  These numbers are taken from a series of microbenchmarks that attempt to 
  test idiomatic usages of the languages, and thus can be viewed as a comparison
  between how fast languages perform when used naturally.\cite{bench}
  \caption{Rust vs C Microbenchmarks}
  \label{fig:bench}
\end{figure}

Important in that last statement was the fact that \rust has the 
\textit{potential}
to be as fast as \c, and while it performs in the same order of magnitude, its not
quite as fast in many cases, as seen in Figure \ref{fig:bench}.
A potential reason this is that \rust doesn't have the compiler maturity \c has.

The exciting part about the future of \rust is not just that it will eventually
reach parity with \c, but that it could eventually go beyond it. There are
already a few \rust specific optimizations that \c and \cpp cannot do, but that
set is far too small to provide significant gains as of yet.

One significant addition, an addition which is the primary subject of this
paper, would be to allow for the primary \rust compiler, \rustc, to split
structures in ways that maximize cache performance through access locality.

Most modern caches have cache lines that are 64 bytes long. Since accesses from
memory are so expensive it is important that as much useful data is in that 64
bytes as possible. When iterating over some array of integers or floating point
numbers each memory access will result in cache lines that contain mostly
interesting data. However, when iterating over some array of structures,
the mostly useful only
if every field of said structure is accessed at the same time. If only part of
the structure is accessed then the remaining fields are just taking up space in
the cache.

A solution to this problem is splitting the structure into several component
parts. Thus, assuming the structure is split such that fields that are accessed
together are together, cache performance should improve since more useful 
information can fit on each line.

Performing this optimization, however, requires some work. Firstly one
needs to decide which fields should go into
each substructure. This is not a trivial problem, but thankfully a decent
solution has been found\cite{Zhong:2004:ARS:996893.996872}.
Essentially running a program one and collecting
a memory trace one can generate a fairly good estimate of which fields are more
often accessed together by measuring reuse distances and 
thus generate a splitting which, while not provably optimal, 
does manage to successfully split structures and
provide locality benefit.

While the authors created a great system for automating the structure splitting 
process they did not find a great way of using said split structures. They did
use the system in a small research compiler for \c, but it was severely limited
by the fact that they had to limit their compiler to a subset of \c programs
which had a fully static type system, had a decent amount of run time overhead, 
and would not have worked in parallel programs without significant work.

\c isn't the only language which has problems with structure splitting
however. It only makes sense for languages whose structures are contiguous
in memory in the first place, preventing languages such as \texttt{Haskell}.
from implementing it.  Additionally,\texttt{Java} and 
other language with inheritance also
have issues, specifically when dealing with class hierarchies. Structure
splitting would be limited to one class in each class hierarchy. This would
make automated splitting a much more difficult problem, as well as 
make user-defined splitting potentially more confusing.

Given that structure splitting only can work well in a language with
a strongly static type system with contiguous structures and no inheritance,
it seems that \rust is one of the few general purpose languages in which
it would be possible to implement proper structure splitting.

\subsection{\rustc}
\label{sec:rustc}

\rustc is the most often used, and essentially only, compiler for \rust.
\rustc is effectively a series of conversions from
\rust code to various internal representations and finally into machine code.
First is tokenizes and converts to an \texttt{AST}, then converts that into a
second \texttt{AST}, called \hir. Before April of 2016, when \mir was created,
\hir was the only
internal representation in which serious semantic analysis was performed, which
is likely part of the reason why very little \rust specific optimizations had
been attempted, since \texttt{AST}'s are typically not suited for serious
optimization work. Instead, optimizations were delegated to \llvmir, which \hir
was translated into.

However, the translation from \hir to \llvmir was a long and complicated process.
Much of the semantics of \rust were encoded in the process. To make this 
process easier, a second internal representation was created, called \mir.	


\subsection{\mir}
\mir is fairly new; it was released in April 2016\cite{mirintro}. It acts as a
intermediary between the second of \rust's AST representations, known as \hir,
and \llvmir.

\mir is a fairly standard intermediate representation. \mir represents functions, 
or function-like units like static blocks and closures, 
as a set of basic blocks. Basic blocks are composed of a series of statements
and end in some sort of terminator, of which there are several types. 
Function calls can only be made as part of a terminator. This is
so it can be known statically that every statement of a basic block will
execute\footnote{This isn't strictly true since inline assembly is evaluated as
  a statement, but inline assembly can only be used during unsafe blocks
  anyways, so its not usually relevant}.

  Since \mir is so new, it fairly easy to verify that structure splitting has
  not been successfully implemented in the language\footnote{In fact, field
  reordering has only recently been implemented, meaning its even less
  likely it has been attempted before\todo[inline]{Find a citation for this}}

\section{Design}
\todo[inline]{Switch to just defining equivalent implementation in macro generation.}

\name (STructure-splitting Optimizations for rUsT) is a structure 
splitting compiler modification that attempts to fix the issues with
previous approaches by implementing structure splitting 
mechanism in \rustc, where static type
safety and general memory safety allow for 
structure splitting with little effect on program
semantics
\footnote{Exactly what effect it has is discussed in the next subsection and section \ref{sec:limits}}.

\name can be roughly separated into 
two parts, 
structure splitting
and \mir modification

\subsection{Structure Splitting}

\begin{figure}[h]
\begin{minted}{rust}
//This annotation...
#[affinity_groups(a = 1, b = 2, c = 1)]
struct S {
  pub a: usize,
  pub b: usize,
  pub c: usize,
}

//...results in the following structures
struct S1 {
  pub a: usize,
  pub c: usize,
}
struct S2 {
  pub b: usize,
}
\end{minted}
  \caption{Structure Splitting}
  \label{fig:split}
\end{figure}

Splitting \rust structures is the simplest part of the process. It is currently
implemented as part of the \rust macro system, where the user of \name
would add an annotation which generates a number
of substructures, each of which contains some subset of it's parent's 
fields. Which fields go in which substructures is decided by arguments 
in the annotations 
this means
that users would need to specify manually how the split the structure. While an
automated solution is of course possible, it is important for a few reasons that
the process be manual, at least at the moment. An example of this
is in Listing \ref{fig:split}

First, deciding how to split a structure is already a solved problem,
but the process is rather expensive. 
It seems like a better idea to just 
allow the user to either; one, 
intelligently reason about there program and split their structure 
how it makes the most sense, or two, 
let them build an already existing tracing solution into their build system,
and thus get the advantage of deciding when and where to spend the 
computation time.

Second, it is important that structure splitting be an ``opt-in'' process.
While \rust is by default a memory safe language, it has an
unsafety system in which memory layout of structures could theoretically matter.
Reasoning about unsafety is currently very difficult in \mir, as
which regions of code are unsafe is not preserved from the transition
from \hir. Also, since it is impossible for a structure splitting scheme
to be provably ideal, leaving a hook for the user to modify is important.

It is important as part of the structure splitting process to also
define a trait \texttt{Splittable} and then implement that
trait for each tuple of substructures.
\texttt{Splittable} is used to essentially redefine
the methods of array, slice, and \vec 
to allow them to be called on tuples of split arrays, slices, and \vec's.  
These redefinitions are mostly minimal and obvious; 
\texttt{push()} pushes the appropriate structure onto the appropriate
vector, \texttt{iter()} calls \texttt{iter()} on all
elements in the tuple and zips them together, and others.
This process is extensible to other data structures. Users can
define their own implementation of splittable, but they
are responcible for making sure that the implementation is correct.
\name checks to see if a function has been redefined in this
way, and replaces it as needed as part of Section \ref{sec:funcall}.

We also want to define a tuple of reference of substructures 
to be coercible to a reference of tuple of references. Why this is 
important is relavent later, but basically
we want the ability to pass a tuple of references 
of structures to a method which expects a reference to an 
unsplit structure. This isn't completely necissarily but 
saves us having to contruct a reference to a tuple by hand.
This is done by implementing \texttt{AsRef} and \texttt{AsMut}
for a structure defined as a tuple of structures.

\subsection{MIR Modification}

The second part of \name is a series of modifications to \mir that actually
implement the structure splitting, the first of which is modifications performed on
local variable declarations.

\subsubsubsection{Declaration Modification}
In order to properly preserve locality gains, certain declarations
we could not otherwise split need to be modified, otherwise we would need
to write back all our split structures into an original structure to
do anything interesting.

One place we unconditionally perform in place modification is in function
arguments. Any time we are in a function that takes in some splittable
argument \texttt{S}, we want to modify the function to take 
in a tuplified version of \texttt{S}, \texttt{(S1, S2, ...)}. 
We tuplify here as primarily a simplification measure, since 
we could equivalently take more arguments, but currently
\mir makes this process difficult and
not tuplifying does make it possible to accidentally make a function
take in more arguments then \rustc allows.
We also tuplify function pointers to ensure our program typechecks.

Another place we do in place modification is as part of tuples. If we have some
tuple which contains a splittable structure we replace the type of that
structure with a tuple containing the substructures of that structure, as 
in Listing \ref{fig:tupletuplify}.

\begin{figure*}[p]
  \begin{minipage}[t]{0.5\linewidth}
\begin{minted}{rust}
let _1: (&S, int, int);
\end{minted}
  \end{minipage}
  \begin{minipage}[t]{0.5\linewidth}
\begin{minted}{rust}
let _1: ((&S1, &S2, ...), int, int);
\end{minted}
  \end{minipage}
  \caption{Tuple Modification}
  \label{fig:tupletuplify}
\end{figure*}

We must do tuplification in type parameters to other structures,
as in Listing \ref{fig:localdeclbefore}.
This guarantees that we use tuplified structures
when the code specifies the usage of the original strucuture
\footnote{This won't work when \rust gets specialization, 
but until then this is safe}.
Doing this is very important as it transforms iterators
of type \texttt{Iterator<S>} into type \texttt{Iterator<(S1, S2, ...)>}

The last type that needs tuplification are function pointers. We 
need to change the parameters and return types to tuplified form
so that our program will typecheck. Since we do not allow our split structure
to escape crate scope, we know that if a function has a splittable
argument \texttt{S}, either it is in some way parameterized with
\texttt{S} and we replace the parameter with a tuplified form
\texttt{(S1, S2, ..)}, or it is a function in our crate, which
means we have already modified it to accept a tuplified form.

\begin{figure}[p]
  \begin{minipage}[t]{0.5\linewidth}
\begin{minted}{rust}
let _1: T<S>;
\end{minted}
  \end{minipage}
  \begin{minipage}[t]{0.5\linewidth}
\begin{minted}{rust}
  let _1: T<(S1, S2, S3)>;
\end{minted}
  \end{minipage}
  \caption{Structure Argument Tuplification}
  \label{fig:localdeclbefore}
\end{figure}

\begin{figure}
  \begin{minipage}[t]{0.5\linewidth}
\begin{minted}{rust}
let _1: S;
\end{minted}
  \end{minipage}
  \begin{minipage}[t]{0.5\linewidth}
\begin{minted}{rust}

let _1: S1;
let _2: S2;
let _3: S3;

\end{minted}
  \end{minipage}
  
  \caption{Reference and Array Splitting $g = 3$}
  \label{fig:localdeclbefore}
\end{figure}
\begin{figure}
  \begin{minipage}[t]{0.5\linewidth}
\begin{minted}{rust}
let _1: &S;
\end{minted}
  \end{minipage}
  \begin{minipage}[t]{0.5\linewidth}
\begin{minted}{rust}
let _1: &S1;
let _2: &S2;
\end{minted}
  \end{minipage}
  \caption{Reference Splitting}
  \label{fig:refedecl}
\end{figure}

\subsubsubsection{Splitting}
\label{sec:splitting}
\begin{figure*}
  \begin{minipage}[t]{0.5\linewidth}
    \begin{minted}{rust}
// reference
let _1: &S;


// unsized array (slice)
let _2: [S];


// sized array
let _3: [S, 10000];


// sized array of references
let _4: [&S, 10000];


\end{minted}
\end{minipage}
\begin{minipage}[t]{0.5\linewidth}
\begin{minted}{rust}
let _1: &S1;
let _2: &S2;
let _3: &S3;

let _4: [S1];
let _5: [S2];
let _6: [S3];

let _7: [S1, 10000];
let _8: [S2, 10000];
let _9: [S3, 10000];

let _10: [&S1, 10000];
let _11: [&S2, 10000];
let _12: [&S3, 10000];
\end{minted}
\end{minipage}


  \caption{Reference and Array Splitting $g = 3$}
  \label{fig:refedecl}
\end{figure*}
Splitting variable declarations is important since it is where
we get our locality benefits.  The simplest case is the one in which
we find a declaration whose type is one of our split structures. In this 
case we can split fairly easily, replacing the declaration of the structure with
declarations for each substructure.  
\footnote{This isn't exactly how it works in the code, instead of 
  \textit{replacing} the declaration we simply add the 
  substructure declarations.  This makes the code simpler, 
  but the detail is largely irrelavent to the
  idea, and thus examples will all assume the original declaration is removed.
  It is removed when the translation to \llvmir is completed anyway.}

Reference, arrays, slices, and vectors are also easy. 
If we see a variable that is reference to a structure 
that we can split, we just split that variable keeping the original
types around each substructure, 
one for each substructure, as in Listing \ref{fig:refedecl}.

\subsubsection{Assignment Modification}
\label{sec:assign}

Now that we have properly typed variables, we need to replace the assignments
in each basic block in order to properly use them. The only time we need
to actually replace an assignment is when it involves a variable which was
split.  
Most of the time if we see an assignment containing a split structure
we just duplicate the assignment, replacing the variables with the proper 
split variables so that the correct information is in the
correct place. The only place we do something different is when
a field of the split structure was access. In that case we 
just need to replace the declaration with the declaration
that contains the needed field.

\subsubsection{Function Call Modification}
\label{sec:funcall}

Modifying function calls is a little more tricky. 

\subsection{Limitations}
\label{sec:limits}
Currently \name does not support defining another structure
which contains a split structure, since the \rustc macro system
currently does not allow modification of other parts of the 
crate. This is of course implementable, but requires extensive
compiler modification and thus was not attempted.

\name also doesn't support exporting split structures outside of the 
current crate.
This is because the compiler cannot easily access code out of the crate, and
thus cannot modify function definitions to accept the tuple of structures we
pass in. This problem could be solved by a few methods, all of which require
either significant compiler modification or the introduction of run
time cost. 

\todo[inline]{Should do this}
\name doesn't split static assignments, primarily since writing to a static
assignment at run time is discouraged.

\section{Preliminary Results}
Due to a combination of factors, a fully automated system for \name could not
be completed in time. As such, evaluation of a full implementation
was not performed. Some test cases were tested by a limited automated implementation,
but others were tested by manually contructing the result. Those such cases
will be marked explicitly, and as implementation continues they should be filled in.



\section{Discussion}
\label{sec:discuss}

This project is very much more of a platform for further research then a
standalone project. At the time of writing simple structure splitting
with arrays and references is all that is completed, and 
while I was successful is performing locality based
optimizations in that specific case without the usual 
run time overhead typical of structure splitting
work, there is considerably more work to be done until I 
would consider this
a project that should be usable by the average user.

\todo[inline]{Talk about applications on locks}
\todo[inline]{Maybe talk about why not \llvm }

\subsection{Unsafety}
\label{sec:unsafe}

My implementation makes the implicit assumption that the code
does not care about the memory layout. While safe 
\rust is guarenteed not to, unsafe \rust can. Unsafe \rust
allows for casting and pointer arithmatic, but \rust
does seem to reserve the right to reorder fields \ref{reorder}.
It is impossible to say however, since \rust doesn't have a
formal language standard, much less a standard for unsafe code.


\subsection{Rust Culture}
\label{sec:culture}

The fact that \rust doesn't have a formal standard yet should not be shocking
for anyone that follows the language. The \rust community has a tendency
to eskew formal documentation on the inner workings of the language.
The only thing close to documentation for \mir is the RFC it was originally
specified in \ref{mirrfc}. However, \mir has changed significantly since then 
and the RFC was not kept up to date. This resulted in a significant time
sink.

\section{Research Platform}

While \name is complete and functional in itself, it opens the door for 
a good deal more research to occur in this domain. Increasing the automation
of the process is definitely important regardless of my hesitence to automate
the structure splitting process. 


% \section{Notes}
% \label{sec:remove}

% \subsection{Justification}
% \label{sec:just}

% \paragraph{Premonomorphization} Helpful for several reasons
% \begin{enumerate}
% \item Could lower compilation time by not having to regenerate code
% \item  Easier to process in our particular case
% \end{enumerate}

% \paragraph{Shorter Compilation Times} One of the biggest sticking points for
% Rust is that compilation times are quite long, in part due to LLVM taking
% significant time in the compilation process (citation needed). Rust's safety
% could potentially allow for excellent incremental compilation down to the basic
% block level (citation needed). Additionally, LLVM takes a significant part of
% Rust compilation times, and moving some optimization up from LLVM to Rust means
% that other, faster backends can become for feasible. LLVM is also not
% necissarily the best at compiling for certain architectures, such as WebASM.
% With Rust getting WebASM as an alternate backend at some point, developing Rust
% specific optimizations could be good.

% %https://github.com/rust-lang/rust/issues/33205

% \subsection{Rvalue Type induction}
% Need to update the \texttt{AggregateKind::Array} so that it references the right Ty

% \section{Challenges}
% \label{sec:annoying}

% \subsection{Deriving}
% Most seem to be ok, but some combinations of derivations make things really
% annoying, ie \texttt{Copy} and \texttt{Clone}.

% \subsection{Polymorphism}

\bibliography{main}{}
\bibliographystyle{acm}
\end{document}
