\documentclass[12pt,oneside]{book}
\usepackage[utf8]{inputenc}

\usepackage{fancyhdr}           % For page number in the upper right
                                % (required) and other running headers
                                % (optional)
\usepackage{setspace}           % For double-spacing (required)
\usepackage{titlesec}           % For keeping section/\chapter titles
                                % single-spaced

\usepackage{blindtext}          % For generating dummy text in the sample
                                % output, can be removed

% Header height (to avoid fancyhdr error)
\setlength{\headheight}{13.6pt}

% Header formatting for regular pages
\fancyhf{}
\fancyhead[L]{\it\small\leftmark}
\fancyhead[R]{\small\thepage}

% Header formatting for \chapter title pages
\fancypagestyle{plain}{%
  \fancyhf{}
  \fancyhead[R]{\small\thepage}
  \renewcommand{\headrulewidth}{0pt}
}

% Formatting of \chapter and section titles: keep them single-spaced in the
% midst of double-spaced text
\titleformat{\chapter}[display]%
{\bfseries\singlespacing\Large}%
{Chapter~\thechapter}%
{1em}%
{\LARGE}

\titleformat{\section}[hang]%
{\bfseries\singlespacing\Large}%
{\thesection}%
{0.25in}%
{}

\titleformat{\subsection}[hang]%
{\bfseries\singlespacing\large}%
{\thesubsection}%
{0.25in}%
{}

\usepackage{listing}
\usepackage{minted}
\usepackage{xspace}
\usepackage[colorinlistoftodos]{todonotes}
\newcommand{\rustname}{{\texttt{Rust}}}
\def \rust {\rustname{}\xspace}
\newcommand{\rustcname}{{\texttt{rustc}}}
\def \rustc {\rustcname{}\xspace}
\newcommand{\cname}{{\texttt{C}}}
\def \c {\cname{}\xspace}
\newcommand{\cppname}{{\texttt{C++}}}
\def \cpp {\cppname{}\xspace}
\newcommand{\mirname}{{\texttt{MIR}}}
\def \mir {\mirname{}\xspace}
\newcommand{\hirname}{{\texttt{HIR}}}
\def \hir {\hirname{}\xspace}
\newcommand{\llvmname}{{\texttt{LLVM}}}
\def \llvm {\llvmname{}\xspace}
\newcommand{\llvmirname}{{\texttt{LLVMIR}}}
\def \llvmir {\llvmirname{}\xspace}
\newcommand{\mlname}{{\texttt{ML}}}
\def \ml {\mlname{}\xspace}
\newcommand{\vecname}{{\texttt{Vec}}}
\def \vec{\vecname{}\xspace}
\newcommand{\projectname}{{\texttt{LAGER}}}
\def \name{\projectname\xspace}

\begin{document}
\title{\name: Safe Structure Splitting in a General Purpose Language}
\author{Jacob Bisnett}
\maketitle
\pagestyle{fancy}
\pagenumbering{arabic}
\chapter{Introduction}
\doublespacing
\label{sec:intro}

\todo[inline]{Why other languages do not permit safe structure splitting (java: inheritance) (haskell: insufficient control over memory), first structure splitting}
\todo[inline]{Don't call it a platform, start by saying its a compiler}

The programming language \rust is fairly new, but already gained a considerable
following due to its placement as a safe and easy to use alternative to \c and
\cpp. \rust's primary advantage is its robust compile time checks and type
system. Among many of these are:
preventing two mutable references to an object from existing at the same time, 
statically determining when to free objects without reference counting or
garbage collection,
and not allowing references to outlive their objects.

Taken all together, \rust helps the user avoid data races and memory safety
errors by preventing programs that could contain them from compiling.
\rust does allow users to circumvent these rules with ``unsafe'' code,
but this unsafety must be explicitly declared and thus dangerous behavior can be
isolated.

While \rust is not the only language that provides this level of safety, it is
the only popular language that provides this level of safety with little to no overhead,
making it potentially as fast as \c and \cpp while avoiding many of the
pitfalls of the two languages.

\begin{figure}
  \centering
\begin{tabular}{r|c|c}
  Test&Rust&C\\
  k-nucleotide&5.30&6.46\\
  pidigits&1.75&1.73\\
  spectral-norm&2.01&1.98\\
  reverse-complement&0.45&0.42\\
  fasta&1.49&1.32\\
  mandelbrot&1.90&1.64\\
  n-body&13.08&9.56\\
  regex-redux&3.28&1.89\\
  binary-trees&4.27&2.39\\
\end{tabular}

  These numbers are taken from a series of microbenchmarks that attempt to 
  test idiomatic usages of the languages, and thus can be viewed as a comparison
  between how fast languages perform when used naturally.\cite{bench}
  \caption{Rust vs C Microbenchmarks}
  \label{fig:bench}
\end{figure}

Important in that last statement was the fact that \rust has the 
\textit{potential}
to be as fast as \c, and while it performs in the same order of magnitude, its not
quite as fast in many cases, as seen in Figure \ref{fig:bench}.
A potential reason this is that \rust doesn't have the compiler maturity \c has.

The exciting part about the future of \rust is not just that it will eventually
reach parity with \c, but that it could eventually go beyond it. There are
already a few \rust specific optimizations that \c and \cpp cannot do, but that
set is far too small to provide significant gains as of yet.

One significant addition, an addition which is the primary subject of this
paper, would be to allow for the primary \rust compiler, \rustc, to split
structures in ways that maximize cache performance through access locality.

Most modern caches have cache lines that are 64 bytes long. Since accesses from
memory are so expensive it is important that as much useful data is in that 64
bytes as possible. When iterating over some array of integers or floating point
numbers each memory access will result in cache lines that contain mostly
interesting data. However, when iterating over some array of structures,
the mostly useful only
if every field of said structure is accessed at the same time. If only part of
the structure is accessed then the remaining fields are just taking up space in
the cache.

A solution to this problem is splitting the structure into several component
parts. Thus, assuming the structure is split such that fields that are accessed
together are together, cache performance should improve since more useful 
information can fit on each line.

Performing this optimization, however, requires some work. Firstly one
needs to decide where to split the structure, which fields should go in
each substructure. This is not a trivial problem, but thankfully a decent
solution has been found\cite{Zhong:2004:ARS:996893.996872}.
Essentially running a program one and collecting
a memory trace one can generate a fairly good estimate of which fields are more
often accessed together by measuring reuse distances and 
thus generate a splitting which, while not provably optimal, 
does manage to successfully split structures and
provide locality benefit.

While the authors created a great system for automating the structure splitting 
process they did not find a great way of using said split structures. They did
use the system in a small research compiler for \c, but it was severely limited
by the fact that they had to limit their compiler to a subset of \c programs
which had a fully static type system, had a decent amount of run time overhead, 
and would not have worked in parallel programs without significant work.

\c isn't the only language which has problems with structure splitting
however. It only makes sense for languages whose structures are contiguous
in memory in the first place, preventing languages such as \texttt{Haskell}.
from implementing it.  Additionally,\texttt{Java} and 
other language with inheritance also
have issues, specifically when dealing with class hierarchies. Structure
splitting would be limited to one class in each class hierarchy. This would
make automated splitting a much more difficult problem, as well as 
make user-defined splitting potentially more confusing.

Given that structure splitting only can work well in a language with
a strongly static type system with contiguous structures and no inheritance,
it seems that \rust is one of the few general purpose languages in which
it would be possible to implement proper structure splitting.

\section{\rustc}
\label{sec:rustc}

\rustc is effectively a series of conversions from
\rust code to various internal representations and finally into machine code.
First is tokenizes and converts to an \texttt{AST}, then converts that into a
second \texttt{AST}, called \hir. Before April of 2016, when \mir was created,
\hir was the only
internal representation in which serious semantic analysis was performed, which
is likely part of the reason why very little \rust specific optimizations had
been attempted, since \texttt{AST}'s are typically not suited for serious
optimization work. Instead, optimizations were delegated to \llvmir, which \hir
was translated into. 

\todo[inline]{More here}

\subsection{\mir}
\mir is fairly new; it was released in April 2016\cite{mirintro}. It acts as a
intermediary between the second of \rust's AST representations, known as \hir,
and \llvmir.

%this is sort of relevant since it saves on some work I discuss later. Might remove
\mir is a fairly standard intermediate representation. \mir represents functions, 
or function-like units like static blocks and closures, 
as a set of basic blocks. Basic blocks are composed of a series of statements
and end in some sort of terminator, of which there are several types. 
Function calls can only be made as part of a terminator. This is
so it can be known statically that every statement of a basic block will
execute\footnote{This isn't strictly true since inline assembly is evaluated as
  a statement, but inline assembly can only be used during unsafe blocks
  anyways, so its not usually relevant}.
  
  \todo[inline]{More here}

\chapter{\name: Structure Splitting in Rust}

\name is a structure splitting compiler modification that attempts to fix the issues with
previous approaches by implementing structure splitting mechanism in \rustc, where static type
safety and general memory safety allow for structure splitting with little effect on program
semantics\footnote{Exactly what effect it has is discussed in the next section and section \ref{sec:limits}}.

\name can be roughly separated into two parts, structure splitting and
\mir modification.

\section{Structure Splitting}

\begin{figure*}
\begin{minted}{rust}
//This annotation...
#[affinity_groups(a = 1, b = 2, c = 1)]
struct S {
  pub a: usize,
  pub b: usize,
  pub c: usize,
}

//...results in the following structures
struct S1 {
  pub a: usize,
  pub c: usize,
}
struct S2 {
  pub b: usize,
}
\end{minted}
  \caption{Structure Splitting as a macro}
  \label{fig:split}
\end{figure*}

Splitting \rust structures is the simplest part of the process. It is currently
implemented as part of the \rust macro system, where the user of my library
would add an annotation, as in Figure \ref{fig:split}, which generates a number
of smaller structures, each of which contains some subset of it's parent's fields.

Currently this means
that users would need to specify manually how the split the structure. While an
automated solution is of course possible, it is important for a few reasons that
the process be manual, at least at the moment. 

Firstly it was mandatory that the strategy used for structure splitting be modifiable by the user.
Since the structure splitting operation is so global and such a potential performance improvement,
some small change could end up resulting in massive performance gains or losses. Since it is impossible
to come up with a $100\%$ correct automated strategy, allowing a manual override is important, or else
user could potentially prevented from making changes to their code
since it could break the optimization.

Secondly, deciding how to split a structure is already a solved problem,
but the process is rather expensive. It seems like a better idea to just allow the user to either; one, 
intelligently reason about there program and split their structure how it makes the most sense, or two, 
let them build an already existing tracing solution into their build system, and thus get the advantage
of deciding when and where to spend the computation time.

Thirdly, it is important that structure splitting be an ``opt-in" process.
While \rust is by default a memory safe language, \rust has an
unsafety system in which memory layout of structures could theoretically matter.
This is less of a problem since usually when reasoning about structure layouts
compiler directive are used to prevent the structure from having strange
layouts, but having an explicit action to activate structure
splitting is important for the default case. 

\todo[inline]{More here}
\section{\mir modification}

The series of \mir modifications needed to automatically implement structure
splitting code is also made of two parts, variable declaration modification and
assignment modification.

\subsection{Declaration Modification}

Variable declarations can take on one of 22 different types, some of which can
be parameterized by other types. Our action when we see a type can be divided
into three cases, no modification, tuplification, and splitting.

\subsubsection{No Modification}
Some types we do not need to touch. These include primative types such as
integers, floats, and booleans, as well as internal error types, types used in
type inference\footnote{Since \mir occurs after types inference}, and any
parameterized type that doesn't contain any structure we are interested in.

\subsubsection{Tuplification}
For types the can contain one of the split structures, we have two primary
options, the first being what we will call tuplification of the type.

\todo[inline]{Fix this paragraph}
Tuplification is when we take one of our structures \texttt{S} and convert it
into a tuple of substructures \texttt{(S1, S2, ..., Sg}. In general this 
doesn't typically improve locality, and is effectively equivalent to 
field reordering\footnote{In fact this could make locality worse since each
structure would need to be aligned properly, making the tuple possibly
bigger then the sum of it's parts.}. However it is important that we do this to
ensure semantic correctness, and it does help use preserve locality, since we can
do the process preserving the recursive type structure, meaning that
we can convert a reference to a structure \texttt{\&S} and convert it into 
\texttt{\&S1, \&S2, ..., \&Sg}, which preserves the original location of the 
data as opposed to simply reassigning the fields into a \texttt{\&S}.

One place we unconditionally perform in place modification is in function
arguments. This is primarily a simplification measure, since its easier to
tuplify the type of function arguments then to extend the signature of the
function. This additionally makes it impossible to give our function more
arguments then \rust allows\todo{Don't know if this is true}.
Similarly we perform in place tupling in function pointers so that we correctly
call our modified functions.

Another place we do in place modification is as part of tuples. If we have some
tuple which contains a splittable structure we replace the type of that
structure with a tuple containing the substructures of that structure.

\todo[inline]{fix this}
Additionally, we also perform tuplification when we attempt to split
structs that we could not ordinarily split that are parameterized by
one of our structures. This again keeps the original locality if
the struct is passed by reference, as well as allowing use to guarentee
semantic correctness since the the structure expects a tuple, and thus
can interface properly with our transformed code.

% \begin{figure*}
%   \begin{minipage}[t]{0.5\linewidth}
% \begin{minted}{rust}
% // This declaration...
% let _1: S;
% \end{minted}
%   \end{minipage}
%   \begin{minipage}[t]{0.5\linewidth}
% \begin{minted}{rust}
% // ... turns into this
% let _1: S1;
% let _2: S2;
% let _3: S3;
% \end{minted}
%   \end{minipage}
  
%   \caption{Simple Local Declarations Splitting $g = 3$}
%   \label{fig:localdeclbefore}
% \end{figure*}

% \todo[inline]{Need to talk about splitting function arguments}
% Of course deciding whether or not to split a declaration is not a easy as just
% checking if it is exactly what we want.  
% There are, at time of writing, 22 different kinds of types a variable declaration
% can have. For the purposes of this paper, these types can be separated into seven
% groups; primitive, internal, reference, arrays, tuples, structures,
% and function pointers.

\begin{figure*}
  \begin{minipage}[t]{0.5\linewidth}
\begin{minted}{rust}
// reference
let _1: &S;


// unsized array (slice)
let _2: [S];


// sized array
let _3: [S, 10000];

\end{minted}
  \end{minipage}
  \begin{minipage}[t]{0.5\linewidth}
\begin{minted}{rust}

let _1: &S1;
let _2: &S2;
let _3: &S3;

let _4: [S1];
let _5: [S2];
let _6: [S3];

let _7: [S1, 10000];
let _8: [S2, 10000];
let _9: [S3, 10000];
\end{minted}
  \end{minipage}
  
  \caption{Reference and Array Splitting $g = 3$}
  \label{fig:refedecl}
\end{figure*}

% We can safely ignore primitive and internal types. References, and
% arrays are split by checking by recursively checking their referent types
% and splitting them if they are splittable, as in Figure \ref{fig:refedecl}.

% Tuples are where decisions start to become complicated, because splitting tuples 
% can lead to some unacceptable run time overheads. If we just split the tuple as we did the
% reference and array cases, where we just create $g$ tuples for each substructure, we are 
% wasting a good deal of time and memory reproducing the assignments to the other elements
% in the tuple. 

% As long as we do not split a tuple with more then one structure, tuples are
% also the same. Why we can't split multiple structures in a tuple is discussed in
% Section \ref{sec:limits}.

\subsection{Structures}
Structures are very easy in the case that the structure is exactly what we are
looking for. Where things get tricky is when some other structure is
parameterized by the structure we are interested in. This case is very important
to handle correctly, since generic structures are very common in \rust. The most common,
and thankfully the easiest to deal with, is \vec. \vec is the \rust structure analygous
to the \cpp \texttt{std::vector}. It is used practically everywhere, properly handling it
is extremely important, since most of the applications we would be targeting would be storing
their structures in a \vec, and thus not handling would result in structure splitting being
rather ineffective.

The current method of handling this is a whitelist of some of the most common containers
in the \rust standard library, with the ability for users to declare their container as split-friendly.
Benefits of this will be discussed in Section \ref{sec:discuss}.

\subsection{Function Pointers}
Functions pointers aren't split in the same way that other types might be, but
we still need to modify their type signature so that we can call them in a way
that preserves our structure splitting benefits. Essentially we walk through the
signature of the function and replace each splittable structure with a tuple
of each of it's substructures. This works because we walk through each argument and 
return value \textit{recursively}, which means that if the function accepts a reference
to a splittable structure we modify it so it takes a tuple of references to substructures.
This is important because iteration in \rust often uses the function \texttt{map}, which
takes in an anonymous function whose arguments are references. 

We are only modifying the signature of the function here, in Section \ref{sec:assign} we will
discuss how we modify the contents of the function in order to actually perform structure
splitting.

\todo[inline]{Finish this}

\todo[inline]{Talk about functions}

\section{Assignment replacement}
\label{sec:assign}
All we have done so far is split variable declarations, we still need

\section{Limitations}
\label{sec:limits}
Currently my library does not support nesting split structures inside other structures.
This is completely feasible, but requires some fairly extensive compiler
modifications in order to work correctly. Many use cases for this functionality
could switch to a tuple based approach.

It also doesn't support exporting split structures outside of the current crate.
This is because the compiler cannot easily access code out of the crate, and
thus cannot modify function definitions to accept the tuple of structures we
pass in. This problem could be solved by a few methods, all of which require
either significant compiler modification or the introduction of run
time cost. 

\chapter{Discussion}
\label{sec:discuss}

This project is very much more of a platform for further research then a
standalone project. At the time of writing simple structure splitting
with arrays and references is all that is completed, and 
while I was successful is performing locality based
optimizations in that specific case without the usual run time overhead typical of structure splitting
work, there is considerably more work to be done until I would consider this
a project that should be usable by the average user.

% \begin{figure*}
%  \begin{minted}[autogobble]{rust}
%     struct Mir {
%         basic_blocks: Vec<BasicBlockData>,
%         // ...
%     }

%     struct BasicBlockData {
%         statements: Vec<Statement>,
%         terminator: Terminator,
%         // ...
%     }

%     struct Statement {
%         lvalue: Lvalue,
%         rvalue: Rvalue
%     }

%     enum Terminator {
%         Goto { target: BasicBlock },
%         If {
%             cond: Operand,
%             targets: [BasicBlock; 2]
%         },
%         // ...
%     }
% \end{minted} 
%   \caption{Vec Splitting}
%   \label{fig:vec-split}
% \end{figure*}

% \todo[inline]{Getting this to work with Vec is actually really important so I might do that this week}
% The first, and most important, requirement would be to allow for the splitting
% of certain types of user defined structures. The first, and most obvious,
% benefit for this would be for splitting various containers, the most obvious
% being \vec. \vec is the default \rust resizable array class, analagous to \texttt{std::vector}.

\todo[inline]{Talk about applications on locks}
\todo[inline]{Maybe talk about why not \llvm or \texttt{ghc}}

\section{Rust Culture}
\label{sec:culture}

Rust has no good
formal specification. The specification for \mir are practically non-existent. %https://github.com/rust-specification/english-specification
The closest thing there is are an RFC proposing \mir, and a blog post announcing
it, neither of which specify exactly what it can do, and both of which are now
out of date with the current implementation.

Thus, until \mir stablizes and \rust gets an official standard, it is impossible
to truly know if this approach will work on all platforms in all cases. As of 

\chapter{Issues}

While in common cases this approach works very well, there are a few issues with
it, most of which resulting from the fact that Rust does not have a languages
standard. As such, it is difficult to be certain that the compiler pass will
have no semantic effect on the code. It works within the current Rust
implementation, but manual testing can only be so successful.
% possibly cite the commit hash?


\section{Unsafety}
\label{sec:unsafe}


Another major issue is that, because there is no language standard, there is no
good way of dealing with unsafe code. Unsafe code allows for raw pointer
arithmetic, which breaks structure splitting. Rust does have the benefit that
unsafe code must be contained in explicitly declared code blocks.

\chapter{Research Platform}

While \name is complete and functional in itself, it opens the door for 
a good deal more research to occur in this domain. Increasing the automation
of the process is definitely important regardless of my hesitence to automate
the structure splitting process. Of considerable importance is also
the ability to decide which structures can be considered containers and thus 
can be split completely as opposed to just tuplified.


% \chapter{Notes}
% \label{sec:remove}

% \section{Justification}
% \label{sec:just}

% \paragraph{Premonomorphization} Helpful for several reasons
% \begin{enumerate}
% \item Could lower compilation time by not having to regenerate code
% \item  Easier to process in our particular case
% \end{enumerate}

% \paragraph{Shorter Compilation Times} One of the biggest sticking points for
% Rust is that compilation times are quite long, in part due to LLVM taking
% significant time in the compilation process (citation needed). Rust's safety
% could potentially allow for excellent incremental compilation down to the basic
% block level (citation needed). Additionally, LLVM takes a significant part of
% Rust compilation times, and moving some optimization up from LLVM to Rust means
% that other, faster backends can become for feasible. LLVM is also not
% necissarily the best at compiling for certain architectures, such as WebASM.
% With Rust getting WebASM as an alternate backend at some point, developing Rust
% specific optimizations could be good.

% %https://github.com/rust-lang/rust/issues/33205

% \section{Rvalue Type induction}
% Need to update the \texttt{AggregateKind::Array} so that it references the right Ty

% \chapter{Challenges}
% \label{sec:annoying}

% \section{Deriving}
% Most seem to be ok, but some combinations of derivations make things really
% annoying, ie \texttt{Copy} and \texttt{Clone}.

% \section{Polymorphism}

\bibliography{main}{}
\bibliographystyle{acm}
\end{document}
%%% Local Variables:
%%% coding: utf-8
%%% mode: latex
%%% TeX-engine: pdflatex
%%% TeX-master: t
%%% TeX-command-extra-options: "-shell-escape"
%%% End:
